\documentclass[11pt,a4paper]{article}

% ---------- Packages ----------
\usepackage[margin=1.2cm]{geometry}
\usepackage{titlesec}
\usepackage{enumitem}
\usepackage{hyperref}
\usepackage{fontspec}
\usepackage{setspace}

% ---------- Fonts ----------
\setmainfont{Times New Roman}

% ---------- Section Style ----------
\titleformat{\section}
  {\large\bfseries}
  {}
  {0em}
  {}
  [\titlerule]

\titlespacing*{\section}{0pt}{8pt}{6pt}

% ---------- List Style ----------
\setlist[itemize]{noitemsep, topsep=2pt, leftmargin=1.2em}

% ---------- Hyperlink ----------
\hypersetup{
  colorlinks=true,
  urlcolor=black
}

% ---------- Document ----------
\begin{document}
\pagenumbering{gobble}

% ---------- Header ----------
\begin{center}
    {\LARGE \textbf{Rui Shao}} \\[6pt]
    +86-15623713170 / +81-07090122378 \quad
    \href{mailto:sr1054461216@gmail.com}{sr1054461216@gmail.com} \quad
    \href{https://github.com/shaorui0}{github.com/shaorui0} \quad

    \end{center}

% ---------- About ----------
\section*{About}

Infrastructure engineer working on large-scale production systems on AWS (10+ regions) and Kubernetes (25+ clusters with 600+ nodes), with a focus on safe deployment, traffic control, observability, and cost-aware operations, increasingly leveraging AI-assisted agents to support decision-making and operational workflows.\\
\noindent Built platforms at the boundary of infrastructure and backend systems, combining automation and AI-driven analysis to make complex environments easier to operate, safer to change, and faster to recover under real production pressure.

% ---------- Experience ----------
\section*{Experience}
% ===== Datavisor =====
\noindent
\textbf{Datavisor} \hfill \textit{Infrastructure Engineer} \hfill 2025/03 -- Present (China, Japan)

\vspace{4pt}

\noindent\textbf{\textit{Platform Control \& Production Infrastructure}}
\begin{itemize}
  \item Built backend control-plane services to manage \textbf{deployments, releases, traffic switching, and access governance} across multi-cloud, multi-cluster environments.
  \item Designed end-to-end change execution workflows with rollback and approval gates, reducing deployment time from \textbf{\textasciitilde30 minutes to \textasciitilde5 minutes} while improving release reliability.
  \item Implemented cross-cluster traffic switching and failover supporting canary and blue--green releases, reducing incident recovery time by \textbf{\textasciitilde80\%}.
  \item Operated and evolved production Kubernetes platforms on AWS, including cluster upgrades, node lifecycle management, and multi-AZ / multi-region architectures with defined RTO/RPO targets.
\end{itemize}

\vspace{3pt}

\noindent\textbf{\textit{Observability, On-call \& Reliability}}

\textit{Service Health, Incident Response}
\begin{itemize}
  \item Built and operated observability systems spanning infrastructure, middleware, and service-level signals across multiple Kubernetes clusters.
  \item Shifted monitoring from resource-centric dashboards to \textbf{service impact and SLO-driven views}, improving signal quality and reducing alert fatigue.
  \item Developed an \textbf{AI-assisted on-call investigation workflow} that aggregates anomalies, affected services/customers, dependencies, recent changes, and logs into a unified context, enabling faster human-led triage and informed incident response, capacity planning, and performance tuning.
\end{itemize}

\vspace{3pt}

\noindent\textbf{\textit{Cost \& Operational Efficiency}}
\begin{itemize}
  \item Optimized cloud costs via selective use of Spot and Reserved Instances, achieving \textbf{20--40\% compute cost reduction} without violating service SLOs.
  \item Reduced storage and data transfer costs through lifecycle policies, tiered storage, and topology-aware optimizations.
\end{itemize}

\vspace{6pt} % company-level semantic break

% ===== Intel =====
\noindent
\textbf{Intel} \hfill \textit{Cloud Software Development Engineer} \hfill 2022/06 -- 2025/02 (China)

\vspace{2pt}
\begin{itemize}
  \item Built core \textbf{distributed task execution and data-processing systems} on Kubernetes, forming a strong foundation in \textbf{concurrency control, state consistency, and failure handling} under parallel workloads.
  \item Engineered a \textbf{high-throughput gRPC streaming service}, using profiling-driven optimization (\textbf{pprof}), memory reuse, and async pipelines to cut latency from \textbf{200ms $\rightarrow$ 50ms} and significantly reduce GC and memory pressure.
  \item Designed \textbf{correctness-first scheduling and version update mechanisms} using transactional boundaries and distributed coordination, preventing race conditions and inconsistent intermediate states.
\end{itemize}

% ---------- Skills ----------
\section*{Skills}

\textbf{Programming Languages:} Python, Golang, Java, Shell, SQL \\
\textbf{Tools:} AWS, GCP, Kubernetes, Docker, MySQL, YugaByte, Clickhouse, Redis, Kafka, Elasticsearch, Prometheus, InfluxDB, Grafana, Loki, gRPC, Helm, Git \\
\textbf{Languages:} Chinese, English



% ---------- Education ----------
\section*{Education}

\textbf{Master}, University of Science \& Technology Beijing (Beijing, China) \hfill 2019/09 -- 2022/06 \\
Computer Science and Technology

\vspace{4pt}
\noindent
\textbf{Bachelor}, Wenhua College (Wuhan, China) \hfill 2014/09 -- 2018/06 \\
Electronic Information Engineering

\end{document}
